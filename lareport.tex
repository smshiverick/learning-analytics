The left side of Table 3 shows that, in the regression with the training set, 
previous course failures, course topic, school support, romantic relationship, 
going out with friends, and overall health were negatively correlated with 
student performance, whereas plans for higher education, study time, mother's
education level, internet access, quality of family relations, and father's 
job were positively related to student achievement. The right side of Table 3
shows the final regression model with the testing set. A one-unit change in 
past course failures yielded a -1.57 decrease in the predicted value of 
student performance, holding the effect of other predictor variables constant. 
For students in the math course, there was a -1.04 decrease in predicted 
performance compared to students in the Portugese course, controlling for all 
other factors. Students receiving school support had a -1.35 lower predicted 
final course grade than students who did not receive school support, holding
other factors constant. There was a -0.81 decrease in predicted performance 
for students in a romantic relationship compared to students who were not in 
a relationship . In terms of positive relationships, having plans to pursue 
higher education yielded a 3.21 increase in predicted student performance 
than having no plans for higher education, controlling for the effect of other
independent variables. A one-unit change in weekly study time resulted in a 
0.78 increase in predicted student performance, holding other variables 
constant. In addition, a unit change in mother's level of education was 
associated with a 0.38 increase in predicted student performance, controlling 
for the effect of other independent variables. 

On the training 13 of the 27 predictor variables were statistically 
significant; however, only 6 of the 13 predictors in the testing set were 
significant, which suggests the model was overfit in the training data. 
