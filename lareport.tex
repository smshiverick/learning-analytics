\documentclass[sigconf]{acmart}

\input{format/final}

\begin{document}
  \title{Predictive Models of Student Performance for Data-Driven 
  Learning Analytics}
  \author{Sean M. Shiverick}
  \affiliation{\institution{smshiverick@gmail.com}
  }
\renewcommand{\shortauthors}{S.M. Shiverick}

%%%%%%%%%%%%%%%%%%%%%%%%%%%%%%%%%%%%%%%%%%%%%%%%%%%%%%%%%%%%%%%%%%%%%%%%%%%%%%%%

\begin{abstract}

Analytic tools are useful for detecting patterns in education data and 
providing insights about student performance and learning. This study compared 
six supervised learning algorithms (linear regression, ridge regression, the 
lasso, regression trees, random forests regression, gradient boosted regression) 
and identified features important for predicting student performance. The 
dataset consisted of N=1044 observations from two secondary schools in 
Portugal \cite{cortez08}. Performance was assessed by final grades 
(\textit{range}: 0-20) in two courses, mathematics and Portugese. The models 
were fit to training data with 27 independent variables and evaluated on a 
testing subset. Overall, performance was lower for students in mathematics 
than Portugese. The models selected a similar set of variables as important 
for predicting performance: mother's education level, student plans for higher 
education, and weekly study time were positively related to predicted 
performance, whereas course subject, school educational support, and romantic 
relationships were associated with decreased student performance. The models 
differed in the number, weighting, order and importance given to predictor 
variables. Linear regression provided a model with 13 predictors. Ridge 
regression shrank the coefficient estimates toward zero; the lasso performed 
variables selection for a model with 20 predictors. There was a tradeoff 
between model complexity and interpretability. The single pruned regression 
tree provided a simple, interpretable non-linear model with four features. Random 
forests regression and gradient boosting reduced overfitting, but were 
more difficult to interpret. Advantages and limitations of the different models 
are discussed. Applications for educational data mining (EDM) and learning 
analytics (LA) are considered. 
\footnote{Address correspondence to \textit{smshiverick@gmail.com}.
First draft completed June 7, 2019; final version submitted June 14, 2019.}

\end{abstract}
\keywords{Predictive Modeling, Variable Importance, Learning Analytics}
\maketitle

%%%%%%%%%%%%%%%%%%%%%%%%%%%%%%%%%%%%%%%%%%%%%%%%%%%%%%%%%%%%%%%%%%%%%%%%%%%%%%%%
\section{Introduction}

Education institutions have generated very large amounts of student data in 
recent decades due to dramatic increases in computing speed and processing power 
\cite{daniel15, daniel16}. The development and use of analytic approaches for 
predictive modeling allows researchers and educators to discover patterns in 
data and provide insights about learning for effective decision making. 
Educational data mining (EDM) and learning analytics (LA) are multidisciplinary 
fields at the intersection of learning science, social science, statistics, and 
computer science that leverage big data to understand learning and the 
environments in which it occurs \cite{siemens13, siemensbaker12}. Predictive 
modeling provides useful methods for analyzing the factors that contribute to 
student success and identifying individuals at risk for dropping out. 
Evaluating different predictive models and approaches to feature selection is 
useful for determining which approach is best for predicting student performance. 

%%%%%%%%%%%%%%%%%%%%%%%%%%%%%%%%%%%%%%%%%%%%%%%%%%%%%%%%%%%%%%%%%%%%%%%%%%%%%%%%

Historically, education institutions have tracked student performance, 
dropout, retention, and used analytic tools to identify factors central 
to learning such as persistence and social integration \cite{ferguson12}. 
As extensive education datasets became available for analysis, EDM/LA 
researchers have applied a diverse range of descriptive, correlational, and 
predictive methodologies to discover potentially useful patterns in the data 
for understanding learners and learning in different contexts \cite{siemens11}. 
The fields of EDM and LA both share the goal of using research methods and
predictive analysis to improve student performance and instructional design 
\cite{baker09, ferguson12, lester19}. EDM research has focused more on the 
technical challenges of extracting value from big data in education 
\cite{penaAyala14, romero10}, whereas LA takes a more holistic, education-focused 
approach to learning that seeks to inform and empower instructors and 
learners \cite{lang17, papamitsiou14}. Despite differences in their respective 
origins and emphases, LA and EDM are complementary approaches that use similar 
methodologies. 

%%%%%%%%%%%%%%%%%%%%%%%%%%%%%%%%%%%%%%%%%%%%%%%%%%%%%%%%%%%%%%%%%%%%%%%%%%%%%%%%

Together, EDM and LA represent an ecosystem of techniques for gathering, 
processing, and acting on data to promote learning. The main analytic approaches 
used in these areas include discovery with models (i.e. modeling), similarity 
grouping, relationship mining, content analysis, and social network analysis 
(SNA) \cite{baker09, siemensbaker12}. These procedures facilitate the 
preparation, measurement, and collection of data about learning activities 
for subsequent analysis, interpretation, and reporting. Student characteristics 
are often modeled in terms of domain knowledge, motivation, metacognitive 
abilities (i.e. thinking about thinking), learning strategies, attitudes, and 
affect \cite{papamitsiou14}. Analyses of learner data and patterns identified 
within these data have been directed at predicting learning outcomes, 
recommending resources, and detecting error patterns \cite{verbert12}. The 
output from EDM/LA research has provided insights for various stakeholders, 
including learners, educators, and administrators. This paper focuses on 
predictive modeling of student performance as a form of data-driven 
learning analytics. 

%%%%%%%%%%%%%%%%%%%%%%%%%%%%%%%%%%%%%%%%%%%%%%%%%%%%%%%%%%%%%%%%%%%%%%%%%%%%%%%%

\subsection{Predictive Modeling}

Predictive modeling involves a set of statistical procedures and automated 
processes for extracting knowledge from data \cite{jamesetal13, kuhn13}. Two 
main branches of predictive modeling are supervised learning and unsupervised 
learning. Supervised learning problems involve prediction about a specific 
outcome or target variable (i.e. course grade) when examples of input/output 
pairs are available in the data. If a dataset has no target outcome, 
unsupervised learning methods (e.g. clustering) can reveal structure in 
unlabeled data. Clustering can be used to group individuals based 
on similar learning profiles. In this study, student performance is analyzed
as a supervised learning problem based on final course grades. 

%%%%%%%%%%%%%%%%%%%%%%%%%%%%%%%%%%%%%%%%%%%%%%%%%%%%%%%%%%%%%%%%%%%%%%%%%%%%%%%%

Two main approaches for supervised learning problems are classification and 
regression. For a binary or categorical outcome that is represented as a class 
label (e.g., pass, fail), a classification model will predict which class or 
category that new instances are assigned to. When the target variable to be 
predicted is measured on a continuous scale (e.g. GPA), a regression model 
tests how a set of attributes or features predicts the target outcome. 
Classification is the most commonly used data analytic method for modeling 
students and their behavior and can include methods such as logistic 
regression, support vector machines, naive Bayes, decision trees, and 
neural networks \cite{Lykourentzou09}. 

%%%%%%%%%%%%%%%%%%%%%%%%%%%%%%%%%%%%%%%%%%%%%%%%%%%%%%%%%%%%%%%%%%%%%%%%%%%%%%%%

The present study compares several regression models of student performance 
to identify the set of features that best predict student performance. Each 
model was first trained on a set of input-output pairs and then used 
to make predictions about new observations that were previously set aside. 
Comparing different predictive models can help determine which model is best
for a given problem with the data available \cite{raschka17}. Past empirical 
findings indicate that, in addition to course assessments (i.e. number of
quizzes passed), student engagement and participation in course activities 
are the most influential predictors of final grades 
\cite{papamitsiou14, romero10}. A student's sense of belonging is also 
essential for engagement and improved course satisfaction, which can in 
turn lead to reduced student dropout. 

%%%%%%%%%%%%%%%%%%%%%%%%%%%%%%%%%%%%%%%%%%%%%%%%%%%%%%%%%%%%%%%%%%%%%%%%%%%%%%%%
\subsection{Linear Models} 

\subsubsection{Linear Regression} 
A general assumption of linear regression is that the target outcome can be 
represented as a linear function of the input features. The standard linear 
model describes the relationship between predicted target variable 
(Y) from a set of features ($X_1$ ... $X_p$), including some measure of error 
(equation 1). The predicted value of the target outcome can be thought of as 
the weighted sum of the input features with the weights or coefficients 
(i.e., beta values) indicating the influence of a given feature on the outcome.
Ordinary least squares (OLS) regression miminizes the distance (i.e., error) 
between the predicted values of \textit{Y} and the observed values in the 
dataset. If the number of observations (\textit{n}) is much larger than the 
number of features (\textit{p}), OLS coefficient estimates will have low 
variance and perform well on test observations; however, if the number of 
observations \textit{n} is not much larger than the number of features 
\textit{p}, high variability in the OLS fit can result in overfitting and 
poor prediction on the test observations. For high-dimensional datasets 
(\textit{$p>>n$}), the least squares coefficient estimate breaks down. The 
simple linear model can be improved by using alternative fitting approaches 
that produce better prediction accuracy and model interpretability 
\cite{jamesetal13}. 

\begin{equation}
  \ Y = \beta_0 + \beta_1X_1 + \beta_2X_2 +... + \beta_pX_p + \epsilon
\end{equation}

%%%%%%%%%%%%%%%%%%%%%%%%%%%%%%%%%%%%%%%%%%%%%%%%%%%%%%%%%%%%%%%%%%%%%%%%%%%%%%%%

In many regression analyses, it is often the case that multiple independent 
variables or features will not be correlated with the target outcome. Three 
methods for improving the fit of linear models are: (a) subset selection, 
(b) dimension reduction, and (c) regularization (i.e.., shrinkage). Determining 
which set of features is best for representing the predicted outcome is 
essential for model interpretation. A straightforward approach to feature 
selection is to first conduct a regression including all the independent 
variables and then rerun the regression while excluding non-significant 
variables from the model. Another approach, termed \textit{regularization}, 
includes all \textit{p} predictor variables, but constrains (i.e., regularizes) 
the coefficient estimates of the independent variables by shrinking them 
towards zero. Regularization reduces variablity, which improves accuracy 
on the testing set with a slight increase in bias. Shrinking the coefficient 
estimates of irrelevant features toward zero reduces overfitting and can
aid model interpretation. 

%%%%%%%%%%%%%%%%%%%%%%%%%%%%%%%%%%%%%%%%%%%%%%%%%%%%%%%%%%%%%%%%%%%%%%%%%%%%%%%%

\subsubsection{Ridge Regression: L2 Penalty} 

As with OLS, ridge regression seeks coefficient estimates that fit the data 
well by reducing error, but ridge regression introduces a shrinkage penalty
(L2) that has the effect of shrinking the coefficient estimates towards zero. 
When the tuning parameter (lambda) is set to zero, the shrinkage penalty has 
no effect and ridge regression produces the least squares estimates. As the 
value of lambda increases, the estimated regression coefficients approach zero 
\cite{jamesetal13}. The advantage of ridge regressions over least squares is 
based on the bias-variance tradeoff. As the tuning parameter lambda increases, 
the flexibility of the ridge regression decreases, leading to decreased variance 
but increased bias. Lower variance is associated with reduced overfitting, 
whereas higher bias can lead the model to miss relevant relations between
features and target outputs (underfitting). Ridge regression is often applied 
after standardizing the predictor variables so that they are all on the same 
scale (e.g., \textit{M}=0, \textit{SD}=1). Ridge regression performs well with
high-dimensional datasets (p$>>$n) by trading off a small increase in bias for a 
large decrease in variance. A disadvantage of ridge regression is that, because 
it includes all predictors in the model, the penalty shrinks the coefficients 
toward zero, but does not set any of them exactly to zero. This can create a 
problem for model interpretation when working with a very large number of features. 

%%%%%%%%%%%%%%%%%%%%%%%%%%%%%%%%%%%%%%%%%%%%%%%%%%%%%%%%%%%%%%%%%%%%%%%%%%%%%%%%

\subsubsection{The Lasso: L1 Penalty} 

The lasso and ridge regression have similar formulations, but the lasso has 
a major advantage over ridge regression as it produces simpler, more 
interpretable models based on a subset of features. The lasso uses the L1 
penalty which has the effect of forcing some of the coefficient estimates to be
equal exactly to zero when the tuning parameter lambda is sufficiently large 
\cite{kuhn13}. The lasso performs variable selection and produces sparse models 
based on a subset of features, which are generally easier to interpret than 
ridge regression. The lasso implicitly assumes that a number of the feature 
coefficients or weight truly equal to zero. In general, the lasso performs 
better than ridge regression in situations where a small number of features 
account for most of the variability in the target outcome, and the remaining 
features have coefficients that are very small or equal to zero. By contrast, 
ridge regression performs better when the target is a function 
of a large number of predictors that contribute approximately equally to the
coefficients. Cross-validation is used to determine which value of the lambda 
parameter is optimal. In general, least squares regression (OLS) performs well 
when the number of observations is larger than the number of features ($\textit{n}>>\textit{p}$); however, ridge regression and the lasso are 
preferred when working with a very large number of predictors ($\textit{p}>>\textit{n}$).

%%%%%%%%%%%%%%%%%%%%%%%%%%%%%%%%%%%%%%%%%%%%%%%%%%%%%%%%%%%%%%%%%%%%%%%%%%%%%%%%

\subsection{Non-Linear Models}

\subsubsection{Regression Trees}
Decision tree models are widely used for classification and regression. Tree 
models are built on a hierarchy of \textit{if-else} questions that proceeds 
from a root node as the starting point and continues through a series of 
decisions. Each node in the tree represents either a question or a terminal 
node (i.e.,leaf) that contains the outcome. In constructing the tree, the 
algorithm searches through all possible decisions, or tests, and finds a 
solution that is most informative about the target outcome. The recursive 
branching process of tree based models yields a binary tree of decisions, 
with each node representing a test that considers a single feature. This 
process of recursive partitioning is repeated until each leaf in the decision 
tree contains only a single target. Prediction for a new data point proceeds 
by checking which region of the partition the new point falls into and 
predicting the majority in that feature space. Tree based models require 
little adjustment and are easy to interpret. A drawback is that they can lead 
to very complex models that highly overfit data used to train the model. A 
good strategy for building a regression tree is to grow a very large tree 
and then prune it back to obtain a subtree that provides the lowest test error 
rate. A good way to prevent overfitting is to use pre-pruning to limit 
the maximum depth of the tree. 

%%%%%%%%%%%%%%%%%%%%%%%%%%%%%%%%%%%%%%%%%%%%%%%%%%%%%%%%%%%%%%%%%%%%%%%%%%%%%%%%
\subsubsection{Random Forest Regression}

A random forest is a collection of decision trees that are each slightly 
different, with each tree overfitting the data in a different way. This 
approach reduces overfitting by building many trees and averaging their results. 
Randomness is introduced into the tree building process in two ways: first, 
by drawing a random subset (i.e. bootstrap sample) of the data, and second 
by selecting a random subset of features at each node branch \cite{breiman01}. 
In building the random forest, the user must first decide how many trees to 
build and the algorithm makes different random choices so that each tree is
distinct \cite{muller17, raschka17}. The bootstrapping method repeatedly draws 
random samples of size \textit{n} from the dataset with replacement. The 
decision trees are built on these random samples that are the same size as 
the original data, with some points missing and some data points repeated. 
The algorithm also selects a random subset of \textit{p} features, that are 
repeated separately at each node, so that each decision at the node branch is 
based on a different subset of features. These two processes help ensure that 
all of the decision trees in the random forest are different. 

%%%%%%%%%%%%%%%%%%%%%%%%%%%%%%%%%%%%%%%%%%%%%%%%%%%%%%%%%%%%%%%%%%%%%%%%%%%%%%%%

\subsubsection{Gradient Boosting}

Similar to random forests, gradient boosting is an ensemble approach that 
builds many smaller trees; however, with each new tree the gradient boosting 
algorithm attempts to correct for deficiencies of the current ensemble. 
In contrast to random forests, gradient boosting grows smaller, stubbier 
trees, and goes after bias \cite{jamesetal13, kuhn13}. Gradient boosted 
regression trees use strong prepruning, with shallow trees of a depth of 
one to five. Thus, each tree provides an estimate of part of the data. 
Combining many shallow trees iteratively improves model performance. 
Gradient boosting and random forests perform well on similar tasks and data. 
A common practice is to first construct random forests and then use gradient 
boosting to improve model accuracy \cite{muller17}. 

 %%%%%%%%%%%%%%%%%%%%%%%%%%%%%%%%%%%%%%%%%%%%%%%%%%%%%%%%%%%%%%%%%%%%%%%%%%%%%%%%

\begin{table*}[ht]
  \caption{Variables in the Student Performance Dataset \cite{cortez08}}
  \label{tab:freq}
  \begin{tabular}{ll}
    \toprule
    \textit{Target Variable} &  \\
    \midrule
    Final course grade (0=Lowest, 20=Highest) & G3 \\
    \midrule
    \textit{Predictor Variables} &    \\
    \midrule
    1. Sex (0=Male,1=Female) & SEX  \\  
    2. Age (15, 16, 17, 18, 19+ years) & AGE  \\
    3. Home address type (0=Rural, 1=Urban) & AREA  \\
    4. Family size (0=Three or less, 1=More than three) & FAMSIZE  \\
    5. Parents' cohabitation status (0=Separate, 1=Together) & PARENTS  \\ 
    6. Mother's education (0=None, 1=Primary, 2=Grades 5-9,  3=Secondary, 4=Higher education) & MEDU  \\
    7. Father's education (0=None, 1=Primary, 2=Grades 5-9,  3=Secondary, 4=Higher education) & FEDU  \\
    8. Mother's job (0=At home, 1=Other, 2=Civil Services, 3=Health Care, 4=Teacher) & MJOB  \\
    9. Father's job (0=At home, 1=Other, 2=Civil Services, 3=Health Care, 4=Teacher) & FJOB  \\
    10. Student's guardian (0=Other, 1=Father, 2=Mother) & GUARDIAN  \\
    11. Time from home to school (1=<15 min, 2=15-30 min, 3=30-60 min, 4=>60 min) & TRAVEL  \\
    12. Weekly study time (1=<2 hours, 2=2-5 hours, 3=5-10 hours, 4=>10 hours) & STUDY  \\
    13. Extra educational support (0=No, 1=Tes) & SCHOOLSUP  \\
    14. Family educational support (0=No, 1=Tes) & FAMSUP  \\
    15, Paid extra subject classes (0=No, 1=Tes) & PAID  \\
    16. Extra-curricular activities (0=No, 1=Tes) & ACTIVITIES  \\
    17. Wants to take higher education (0=No, 1=Yes) & HIGHER  \\
    18. Internet access at home (0=No, 1=Tes) & INTERNET  \\
    19. In a romantic relationship (0=No, 1=Tes) & ROMANTIC  \\
    20. Quality of family relationships (1=Very Bad, 5=Excellent) & FAMREL  \\
    21. Free time after school (1=Very Low, 5=Very High) & FREETIME  \\
    22. Going out with friends (1=Very Low, 5=Very High) & GOOUT  \\
    23. Workday alcohol consumption (1=Very Low, 5=Very High) & DALC  \\
    24. Weekend alcohol consumption (1=Very Low, 5=Very High) & WALC  \\
    25. Current health status (1=Very Bad, 5=Very Good) & HEALTH  \\ 
    26. Number of school absences (Count range: 0 to 93) & ABSENCES  \\
    27. Course subject (0=Portugese, 1=Mathematics) & COURSE  \\
    \bottomrule
  \end{tabular}
\end{table*}

%%%%%%%%%%%%%%%%%%%%%%%%%%%%%%%%%%%%%%%%%%%%%%%%%%%%%%%%%%%%%%%%%%%%%%%%%%%%%%%%

\subsection{Study Goals} 

This project examines the relationships between student characteristics,  
behavior, and performance. Data on student performance from two secondary 
schools in Portugal was obtained from the UC-Irvine machine learning 
repository (UCI-MLR) \cite{cortez08}. The dataset included information 
from a student survey and school grade records. The data were fit using 
several supervised learning regression models (described above). The 
predictor variables of interest were demographic features, family 
characteristics, and student behaviors (e.g., weekly study hours, romantic 
relationships). The different models explored various dimensions of student 
performance by: (i) Analyzing the combination of factors that best predict 
student performance, (ii) Selecting the variables most important for 
predicting performance, and (iii) Identifying the most accurate and 
interpretable model of predicted student performance. 

%%%%%%%%%%%%%%%%%%%%%%%%%%%%%%%%%%%%%%%%%%%%%%%%%%%%%%%%%%%%%%%%%%%%%%%%%%%%%%%%

\section{Method}

\subsection{Data}

The student performance dataset downloaded from the UCI-MLR was saved as a 
data frame object in a python interactive notebook. The data was collected 
from two secondary schools in the Alentejo region of Portugal during the 
2005-2006 school year and contained information from a questionnaire and 
school reports of student grades \cite{cortez08}. The sample consisted of 
1044 students (56.6\% female, \textit{M}age=16.71 years, \textit{SD}=1.19,
\textit{Median}=17 years, \textit{range}=15-22). Age was measured as a 
categorical variable (\textit{n}=10 individuals between the ages of 20 to 22 
were included in the category: 19+ years). The dataset consisted of 30 
independent variables, including demographic information, social/ emotional
attributes, school-related variables, and student behaviors (see Table 1). 
The target variable, student performance was evaluated on a 20 point scale
\-\ as in other European countries (e.g. France)\-\ at three points during 
the school year (i.e., Grade1, Grade2, Grade3) for two courses: Mathematics
(\textit{n=395}) and Portugese (\textit{n=649}). The target variable of 
interest was the final course grade (G3). A binary dummy variable of
student performance was calculated based on the measure of final exam grades 
(Pass: \textit{G3}$>$10, Fail: \textit{G3}$<=$10) for descriptive purposes.

%%%%%%%%%%%%%%%%%%%%%%%%%%%%%%%%%%%%%%%%%%%%%%%%%%%%%%%%%%%%%%%%%%%%%%%%%%%%%%%%

\subsection{Model Construction}

\subsubsection{Linear Regression (OLS)} 

All models were constructed in \textit{R} (using \textit{Rstudio})
\cite{jamesetal13}. After preliminary exploration of the data, the sample 
was divided into the training set ($n_1$=731) and testing set ($n_2$=313) 
using a 70 to 30 percent split. Each model was first fit to training data 
and evaluated on the testing set. Student performance was regressed on 
27 independent variables shown in Table 1 using the general linear model 
(OLS). The regression model was run on the training set with the full 
set of predictor variables; the model was then rerun excluding all 
non-significant predictors variables from model. The final model was then 
evaluated on a subset of hold-out data in the testing set. 

%%%%%%%%%%%%%%%%%%%%%%%%%%%%%%%%%%%%%%%%%%%%%%%%%%%%%%%%%%%%%%%%%%%%%%%%%%%%%%%%

\subsubsection{Ridge regression (L2 penalty)} 

The glmnet package was used to fit the ridge regression and lasso models. 
The glmnet() function does not use model formula language, so the \textit{X} 
matrix of predictors and target vector \textit{Y} were passed to the model. 
The model.matrix() function produced a matrix corresponding to the 27 predictors 
and automatically transformed any qualitative variables into dummy variables. 
The \textit{alpha} parameter in the glmnet() function determines what kind of 
model is fit: \textit{alpha} = 0 is used to fit ridge regression. It is 
important to select an appropriate value of the parameter \textit{lambda}, 
as the algorithm generates a different set of coefficients for each value of
lambda. By default, the glmnet() function performs ridge regression for an
automatically selected range of lambda values (e.g. 100). The glmnet function 
also standardizes the variables so they are all on the same scale. The 
shrinkage penalty is applied to every feature, but not the intercept. 

%%%%%%%%%%%%%%%%%%%%%%%%%%%%%%%%%%%%%%%%%%%%%%%%%%%%%%%%%%%%%%%%%%%%%%%%%%%%%%%%

\subsubsection{The Lasso (L1 Penalty)} 
 
The Lasso model was fit using the glmnet() function with alpha=1. The model 
automatically calculates correlation estimates for a wide range of lambda 
values. Cross-validation was used to select an optimal value of the tuning 
parameter lambda. The lasso is similar to best subset selection as it tries 
to find the set of coefficient estimates that leads to the smallest error (RSS). 
In terms of the bias-variance tradeoff, the lasso is qualitatively similar to 
ridge regression. As the value of \textit{lambda} increases, the variance
decreases and bias increases somewhat. 

%%%%%%%%%%%%%%%%%%%%%%%%%%%%%%%%%%%%%%%%%%%%%%%%%%%%%%%%%%%%%%%%%%%%%%%%%%%%%%%%

\subsubsection{Regression Trees} 

The regression tree model of student performance was fit to the training data 
using the rpart() function in R, with all 27 independent variables. The decision
tree uses recursive binary splitting to construct a large tree on the 
training data. Cross-validation was used to determine the optimal tree complexity.
The model was prepruned to a maximum depth of 3, which means the algorithm split 
on three consecutive features.

%%%%%%%%%%%%%%%%%%%%%%%%%%%%%%%%%%%%%%%%%%%%%%%%%%%%%%%%%%%%%%%%%%%%%%%%%%%%%%%%

\subsubsection{Random Forests Regression} 

The random forest model was fit using 1000 trees, with all of the features 
considered at each node to determine the randomness of each tree. In general,
random forests work well without very much parameter tuning or scaling of data. 
The important parameters for the random forests algorithm are the number of 
sampled data points and the maximum number of features; the algorithm can 
look at all of the features in the dataset or a limited number. A high value 
for \emph{maximum-features} will produce trees in the random forest that are 
very similar and will fit the data easily based on the most distinctive features, 
whereas a low value will produce trees that are very different from each other, 
which reduces overfitting. 

%%%%%%%%%%%%%%%%%%%%%%%%%%%%%%%%%%%%%%%%%%%%%%%%%%%%%%%%%%%%%%%%%%%%%%%%%%%%%%%%

\subsubsection{Gradient Boosted Regression Trees} 

In addition to pre-pruning and the number of trees, an important parameter 
for gradient boosting is the \emph{learning rate} which determines how 
strongly each tree tries to correct for mistakes of previous trees. A high 
learning rate produces stronger corrections, allowing for more complex models. 
The gbm package was loaded, and the gbm() function was called on student
performance (final grade) using the Gaussian distribution, with 1000 shallow 
trees, a shrinkage parameter = 0.01, and interaction depth of 4 splits. 

%%%%%%%%%%%%%%%%%%%%%%%%%%%%%%%%%%%%%%%%%%%%%%%%%%%%%%%%%%%%%%%%%%%%%%%%%%%%%%%%

\begin{table}
  \caption{Correlation Matrix of Previous Course Failures and Course Grade 
  Variables}
  \label{tab:freq}
  \begin{tabular}{lllll}
    \toprule
    Variable    & Failures  & Grade 1 & Grade 2 & Grade 3  \\
    \midrule
    Failures    &  1.00     &  0.37***  & 0.38***   &  0.38***  \\
    Grade 1     &           &  1.00     & 0.86***   &  0.81***  \\
    Grade 2     &           &           & 1.00      &  0.91***  \\  
    Grade 3     &           &           &           &  1.00     \\    
    \bottomrule
    Note. *** \textit{p}$<$0.001 & & &
  \end{tabular}
\end{table}

%%%%%%%%%%%%%%%%%%%%%%%%%%%%%%%%%%%%%%%%%%%%%%%%%%%%%%%%%%%%%%%%%%%%%%%%%%%%%%%% 

\begin{figure}[!ht]
  \centering\includegraphics[width=\columnwidth]{images/Figure1.pdf}
  \caption{Proportion of Passing or Failing Final Grades as a 
  Function of Previous Course Failures}
  \label{f:Figure1}
\end{figure} 
 
%%%%%%%%%%%%%%%%%%%%%%%%%%%%%%%%%%%%%%%%%%%%%%%%%%%%%%%%%%%%%%%%%%%%%%%%%%%%%%%% 
 
 
\begin{table}
  \caption{Summary Table of Student Performance by Final Course Grade 
  (Pass$>=$10, Fail$<$10) for Selected Variables}
  \label{tab:freq}
  \begin{tabular}{llllll}
    \toprule
                    &  \textbf{Pass} & & & \textbf{Fail} & \\
    Attribute & \textit{N} & \% &  & N & \% \\
    \midrule
    \textbf{Total}  & 661 & 63.3\% & & 383 & 36.7\% \\
    \midrule
    Male            & 277 & 61.1\% & & 176 & 38.9\%  \\
    Female          & 384 & 65.0\% & & 207 & 35.0\%  \\
    \midrule
    \textbf{Course} &  &  &  &  & \\
    Portugese       & 452 & 69.6\% & & 197 & 30.4\%  \\
    Math            & 209 & 52.9\% & & 186 & 47.1\%  \\  
    \midrule
    \textbf{Mother's Education} &  &  &  &  & \\
    Higher Ed       & 235 & 76.8\% & &  71 & 23.2\% \\
    Secondary       & 143 & 49.5\% & &  95 & 32.9\% \\
    Grades 5 to 9   & 180 & 75.6\% & & 109 & 45.8\% \\
    Primary         &  98 & 47.5\% & & 106 & 52.5\% \\
    None            &   7 & 77.8\% & &   2 & 22.2\% \\
    \midrule   
    \textbf{Higher Education Plans} &  &  &  &  & \\
    Planned         & 640 & 67.0\% & & 315 & 33.0\%  \\
    No Plans        &  21 & 23.6\% & &  68 & 76.4\%  \\
    \midrule
    \textbf{School Support} &  &  &  &  & \\
    Received        &  63 & 52.9\% & &  56 & 47.1\%  \\
    None            & 598 & 64.6\% & & 327 & 33.4\%  \\ 
    \midrule    
    \textbf{Study Time} &   &  &  &  &     \\
    More than 10 hrs. &  45 & 72.6\% & &  17 & 27.4\% \\
    5 to 10 hrs.      & 123 & 75.9\% & &  39 & 24.1\% \\
    2 to 5 hrs.       & 321 & 63.8\% & & 182 & 36.2\% \\    
    Less than 2 hrs.  & 172 & 54.3\% & & 145 & 45.7\% \\
    \midrule
    \textbf{Romantic Relationship} & &  &  &  & \\
    Yes             & 221 & 59.6\% & & 150 & 40.4\%  \\    
    None            & 440 & 65.4\% & & 233 & 34.6\%  \\
    \midrule
    \textbf{Internet Access} & &  &  &  & \\
    Yes             & 543 & 65.7\% & & 284 & 34.3\%  \\    
    None            & 118 & 54.4\% & &  99 & 45.6\%  \\
    \bottomrule
  \end{tabular}
\end{table}

%%%%%%%%%%%%%%%%%%%%%%%%%%%%%%%%%%%%%%%%%%%%%%%%%%%%%%%%%%%%%%%%%%%%%%%%%%%%%%%% 

\begin{table*}[ht]
  \caption{Coefficient Estimates for Regression Models of Student Performance on 
  Training Set and Testing Set}
  \label{tab:freq}
  \begin{tabular}{lllllll}
    \toprule
                        & Training Set      &        &         & Testing Set &         &          \\
    \midrule
    Variables           & Coefficient        & S.E.  & t-value & Coefficient & S.E. & t-Value  \\
    \midrule
    Intercept           &       13.289***    & 2.166 &  6.14  &          6.955*     & 3.629 &  1.92 \\
    Course              & \textbf{-2.225}*** & 0.261 & -8.53  & \textbf{-1.162}***  & 0.445 & -2.61 \\
    Mother's Education  & \textbf{ 0.485}*** & 0.122 &  3.97  & \textbf{ 0.473}**   & 0.213 &  2.25 \\
    Go Out with Friends & \textbf{-0.442}*** & 0.114 & -3.89  &         -0.097      & 0.178 & -0.55 \\
    Higher Ed           & \textbf{ 1.695}*** & 0.487 &  3.48  & \textbf{ 4.101}***  & 0.772 &  5.31 \\    
    School Support      & \textbf{-1.435}*** & 0.416 & -3.45  & \textbf{-1.481}**   & 0.647 & -2.29 \\
    Health              & \textbf{-0.262}*** & 0.088 & -2.97  &         -0.008      & 0.152 & -0.05 \\
    Study Time          & \textbf{ 0.454}*** & 0.156 &  2.91  & \textbf{ 0.893}***  & 0.260 &  3.43 \\
    Internet Access     & \textbf{ 0.819}**  & 0.321 &  2.55  &          0.420      & 0.542 &  0.77 \\
    Family Relations    & \textbf{ 0.340}**  & 0.140 &  2.43  &         -0.096      & 0.215 & -0.45 \\
    Romantic Relation   & \textbf{-0.600}**  & 0.266 & -2.26  & \textbf{-1.127}**   & 0.461 & -2.45 \\
    Age                 & \textbf{-0.250}**  & 0.114 & -2.20  &         -0.034      & 0.191 & -0.18 \\
    Family Size         & \textbf{-0.500}*   & 0.278 & -1.80  &         -0.594      & 0.458 & -1.30 \\
    Father's Job        & \textbf{ 0.278}*   & 0.145 &  1.92  &         -0.050      & 0.245 & -0.20 \\
    \midrule
    \textit{n}          &       730          &       &        &         314         &       &       \\
    \textit{F-Value}    & \textbf{15.41}***  &       &        & \textbf{5.96}***    &       &       \\
    \textit{df}         &   (13, 716)        &       &        &      (13, 300)      &       &       \\
    \textit{$R^2$}      &      0.219         &       &        &         0.205       &       &       \\ 
    \textit{Adj. $R^2$} &      0.204         &       &        &         0.171       &       &       \\
    \textit{Resid. S.E} &      3.396         &       &        &         3.643       &       &       \\
    \bottomrule
    Note. Significance levels & *$<$0.10      & **$<$0.05  & ***$<$0.01 & & &
  \end{tabular}
\end{table*}

%%%%%%%%%%%%%%%%%%%%%%%%%%%%%%%%%%%%%%%%%%%%%%%%%%%%%%%%%%%%%%%%%%%%%%%%%%%%%%%% 

\section{Results}

\subsection{Exploratory Data Analysis}

Preliminary examination of the data revealed significant inter-correlations 
between the course evaluation variables Grades 1, Grade 2, and Grade 3 
which accounted for significant portions of variance in the target outcome 
(see Table 2). In addition, previous course failures was significantly 
correlated with all three course grade measures. To address the issue of 
multicolinearity, Grade 1, Grade 2, and past course failures were not included 
in the regression analyses reported below.

%%%%%%%%%%%%%%%%%%%%%%%%%%%%%%%%%%%%%%%%%%%%%%%%%%%%%%%%%%%%%%%%%%%%%%%%%%%%%%%% 

Table 3 provides descriptive statistics for selected attributes. Chi-squared 
tests of independence were used to compare the proportion of students who 
received passing and failing grades by attribute. There was no relationship
between performance and sex; males and females did not differ significantly 
in performance (\textit{p}$=$0.20). Student performance did vary 
according to mother's level of education (\textit{p}$<$0.05), but 
as seen in Table 3, the relationship between performance and mother's 
education was non-linear. Performance also varied significantly by course 
subject (\textit{p}$<$0.001); more than two-thirds of students in 
the Portugese course successfully passed, whereas just over half of students 
in the mathematics course received a passing grade. The relationship between 
student performance and plans for higher education was significant 
(\textit{p}$<$0.001). Two-thirds of students with plans for 
higher education received a passing grade, whereas less than one-quarter of 
students with no plans for higher education passed their course. Extra 
educational school support was significantly related to performance 
(\textit{p}$<$0.001). Just over half of students who received 
extra educational support at school received a passing grade compared to 
nearly two-thirds of students who did not receive extra support. 

As expected, weekly study time was significantly associated with student 
performance (\textit{p}$<$0.001). The proportion of passing and 
failing grades significantly different for students who studied 5 hours or 
more per week compared to students who studied less than 5 hours per week. 
The association between romantic relationships and student performance 
was marginally significant (\textit{p}$<$0.06). The proportion 
of passing and failing grades was significantly different for students in 
a romantic relationship than students not in a romantic relationship. The 
relation between internet access and student performance was also marginally 
significant (\textit{p}$<$0.06). The proportion of passing and
failing grades was significantly different for students with access to the 
internet at home compared to students without home internet access 
(\textit{p}$<$0.05).

%%%%%%%%%%%%%%%%%%%%%%%%%%%%%%%%%%%%%%%%%%%%%%%%%%%%%%%%%%%%%%%%%%%%%%%%%%%%%%%%

\subsection{Linear Regression and Regularization}

\subsubsection{General Linear Model}

Student performance was first regressed on the 27 independent variables 
(Table 1) with the training set; this regression was statistically 
significant and accounted for 19.7\% of the variance in the predicted
value of student performance, taking into account the number of 
independent variables, \textit{F}(27, 702) = 7.62, \textit{p} $<$ 0.001 
($R^2$=0.227, adjusted $R^2$=0.197). The regression model was rerun, 
excluding the non-significant predictors, and this regression also 
yielded a significant relationship between student performance and the 
independent variables, accounting for 20.4\% of the variability in 
predicted performance, \textit{F}(12, 717) = 21.79, \textit{p} $<$ 0.001 
($R^2$=0.219, adjusted $R^2$=0.204). An ANOVA test showed no significant 
difference between the two models (\textit{F}$<$ 1.0, \textit{p} $=$ 0.91) 
and the simpler model with thirteen predictor variables was retained as the 
final model. The estimated coefficients, standard error, and t-value on the 
testing set (ranked by t-Value) are presented in the left side of Table 3.

%\begin{equation*}
% \begin{aligned}
%   Y = 13.29 - 2.23(Course) - 1.44(SchoolSupp) + 1.70(HighEd) - \\
%   (0.44GoOut) + 0.49(MotherEd) -0.26(Health) + \\
%   0.45(StudyTime) + 0.34(FamilyRelations) - 0.25(Age) - \\
%   0.60(RomanticRel) + 0.82(Internet) - \\
%   0.50(FamilySize) + 0.278(FatherJob) 
% \end{aligned}
%\end{equation*}

%%%%%%%%%%%%%%%%%%%%%%%%%%%%%%%%%%%%%%%%%%%%%%%%%%%%%%%%%%%%%%%%%%%%%%%%%%%%%%%%

The final model was evaluated on the testing set and the regression yielded 
a significant relationship between student performance and the independent 
variables, accounting for 17.1\% of the variability of the predicted value 
of student performance, taking into account the number of independent 
variables, \textit{F}(13, 300) = 5.96, \textit{p} $<$ 0.001 ($R^2$=0.205, 
adjusted $R^2$=0.171). As shown in Table 3, 6 of the 13 independent
variables in the testing set were significant, which suggests that the 
model was overfit to data in the training set, The predicted values of
student performance are be explained by the combined effect, or 
weighted average, of the coefficient estimates and the observed values 
for each significant independent variable in the model (Equation 2).

\begin{equation*}
 \begin{aligned}
  Y = 6.955 - 1.162(Course) + 0.473(MotherEd) + 4.101(HigherEd) \\
  +  0.893(StudyTime) - 1.481(SchoolSupport) - 1.127(RomanticRel) \\
 \end{aligned}
\end{equation*}

%%%%%%%%%%%%%%%%%%%%%%%%%%%%%%%%%%%%%%%%%%%%%%%%%%%%%%%%%%%%%%%%%%%%%%%%%%%%%%%%

On the testing set, there was a 1.16 decrease in predicted performance 
for students in the math course compared to students in the Portugese course, 
controlling for all other independent variables. A unit change in mother's 
level of education was associated with a 0.47 increase in predicted student 
performance, controlling for all other variables. Students with plans to pursue
higher education had a 4.10 higher predicted final grade than students with 
no plans for higher education, controlling for other variables. A one-unit 
change in weekly study time resulted in a 0.89 increase in predicted student 
performance, holding constant the effect of other variables. Students receiving 
school support had a 1.48 lower predicted final course grade than students who 
did not receive school support, holding all other variables constant. Finally, 
there was a -1.13 decrease in predicted performance for students in a romantic 
relationship compared to students not in a romantic relationship, controlling
for all other independent variables. 

%%%%%%%%%%%%%%%%%%%%%%%%%%%%%%%%%%%%%%%%%%%%%%%%%%%%%%%%%%%%%%%%%%%%%%%%%%%%%%%%

\begin{figure}[!ht]
  \centering\includegraphics[width=\columnwidth]{images/Figure2.pdf}
  \caption{Coefficient Estimates for the Ridge Regression model 
  (L2 Penalty) as a function of the log Values of Lambda}
  \label{f:Figure2}
\end{figure}

\begin{figure}[!ht]
  \centering\includegraphics[width=\columnwidth]{images/Figure3.pdf}
  \caption{Coefficient Estimates for the Lasso Regression model 
  (L1 Penalty) as a function of the log Values of Lambda}
  \label{f:Figure3}
\end{figure}

%%%%%%%%%%%%%%%%%%%%%%%%%%%%%%%%%%%%%%%%%%%%%%%%%%%%%%%%%%%%%%%%%%%%%%%%%%%%%%%% 

\begin{table}
  \caption{Coefficient Estimates for Ridge Regression and the Lasso Model of 
  Student Performance using Best Value of Lambda from Cross-Validation}
  \label{tab:freq}
  \begin{tabular}{lll}
    \toprule    
                        &   Ridge (L2 Penalty)  & Lasso (L1 Penalty) \\
    \midrule
    Best lambda (CV)    &    0.819      &  0.071    \\
    MSE                 &   13.779      & 13.895    \\
    \midrule
    \textit{Predictor Variables}  & \textit{Coefficients} &  \textit{Coefficients} \\
    \midrule
    Intercept           &   13.498      &   12.899   \\
    Course              &   -1.801      &   -2.058   \\
    Higher Ed           &    1.390      &    1.534   \\
    School Support      &   -1.148      &   -1.121   \\
    Internet Access     &    0.602      &    0.614   \\
    Romantic Relation   &   -0.487      &   -0.461   \\    
    Family Size         &   -0.447      &   -0.353   \\   
    Study Time          &    0.345      &    0.353   \\   
    Mother's Education  &    0.326      &    0.403   \\    
    Going Out           &   -0.280      &   -0.316   \\
    Family Relations    &    0.274      &    0.247   \\
    Family Support      &   -0.258      &   -0.152   \\ 
    Area (Urban/Rural)  &    0.242      &    0.192   \\
    Age                 &   -0.223      &   -0.196   \\ 
    Father's Job        &    0.209      &    0.205   \\
    Health              &   -0.203      &   -0.199   \\
    Weekly Alcohol Cons.&   -0.130      &   -0.113   \\
    Mother's Job        &    0.127      &    0.066   \\
    Travel Time to School & -0.109      &   -0.053   \\
    Parents Rel. Status &    0.258      &    0.045   \\   
    Daily Alcohol Cons. &   -0.072      &   -0.029   \\ 
    Sex                 &    0.118      &    0.   \\
    Paid Extra Classes  &   -0.050      &    0.   \\      
    Extra Activities    &    0.015      &    0.   \\    
    Free Time           &   -0.062      &    0.   \\    
    Absences            &    0.002      &    0.   \\
    Father's Education  &    0.046      &    0.   \\         
    Student Guardian    &   -0.052      &    0.   \\    

    \bottomrule
  \end{tabular}
\end{table}

%%%%%%%%%%%%%%%%%%%%%%%%%%%%%%%%%%%%%%%%%%%%%%%%%%%%%%%%%%%%%%%%%%%%%%%%%%%%%%%% 

\subsubsection{Ridge Regression (L2 Penalty)}

Figure 2 plots the coefficient estimated from the ridge regression model (L2) 
as a function of the log values of lambda (x-axis). As the values of lambda 
become very large, the model shrinks the coefficient values of non-relevant 
predictor variables towards zero, but the values are never exactly equal to 
zero. Cross-validation was used to obtain the best value of lambda, which 
was, $\lambda$ = 0.819 (ln$\lambda$  = -0.200). As shown in Figure 2, the 
predictors with the highest coefficient values were course subject (27), 
students' plans for higher education (17), extra school support (13), and 
internet access at home (18). The ridge regression (L2) was rerun using the
best value of lambda from cross-validation with all 27 predictor variables 
with coefficient estimates shown in Table 5. The ridge regression model had 
a mean squared error (MSE) of 13.78. 

%%%%%%%%%%%%%%%%%%%%%%%%%%%%%%%%%%%%%%%%%%%%%%%%%%%%%%%%%%%%%%%%%%%%%%%%%%%%%%%%

\subsubsection{The Lasso (L1 Penalty)}

Figure 3 plots the estimated coefficients from the lasso (L1) regression as 
a function of the log value of lambda, with the number of associated features 
listed across the top of the plot. The plot shows that as the values of lambda 
increase, the L2 penalty shrinks many of the coefficient values to be equal 
exactly to zero. Cross-validation was used to obtain the best value of lambda, 
$\lambda$ = 0.071 (ln$\lambda$ = -2.65). The lasso model was rerun using the 
optimal value of lambda selected by cross validation with 20 predictor 
variables; the model had a mean squared error (MSE) of 13.895 and accounted 
for approximately 20 percent of the variability in student performance. 
Similar to the ridge regression, the predictors with the highest coefficients 
were course subject (27), plans for higher education (17), extra school support 
(13), and internet access at home (18). The error from the lasso model is very 
similar to the ridge regression, but the lasso has an advantage over ridge 
regression in that the resulting coefficient estimates are sparse and the 
model selected a subset of the predictor variables. 

%%%%%%%%%%%%%%%%%%%%%%%%%%%%%%%%%%%%%%%%%%%%%%%%%%%%%%%%%%%%%%%%%%%%%%%%%%%%%%%%

\subsection{Decision Tree Models}

\begin{figure}[!ht]
  \centering\includegraphics[width=\columnwidth]{images/Figure4.pdf}
  \caption{Regression Tree Model of Student Performance on the 
  Traing Set (\textit{n}=700)}
  \label{f:Figure4}
\end{figure}

\subsubsection{Regression Trees}

The decision tree model was fit to the training set, with a maximum depth of 3;
Figure 4 shows the resulting regression tree with course subject as the root 
node and 7 terminal nodes. Course subject was a dummy variable; the branch to 
the left represents students in the mathematics course (39\%) and the branch to 
the right represents students in the Portugese course (61\%). In addition to 
course subject, the algorithm split on mother's education level, weekly study 
time. student absences, and age in constructing the tree. The values of student 
performance ranged from 5, for students in the mathematics course with no 
absences who were 18 years or older, to 13 for students in the Portugese course 
whose mother's had some higher education. The regression tree model was evaluated 
on the test set (maximum depth=3) which yielded a tree with plans for higher 
education (rather than course subject) as the root node and 6 terminal nodes 
(see Figure 5). The MSE for the regression tree on the testing set was 16.385.

\begin{figure}[!ht]
  \centering\includegraphics[width=\columnwidth]{images/Figure5.pdf}
  \caption{Regression Tree Model of Student Performance on the
  Testing Set (\textit{n}=314)}
  \label{f:Figure5}
\end{figure}

As seen in Figure 5, from the root node of plans for higher education, the 
algorithm split at nodes for mother's education level, area (urban / rural),
course subject, and student absences in constructing the tree. Following 
the right branch from the root node, students with plans for higher education 
(91\%) had a mean predicted performance of 12, whereas on the left branch, 
students with no plans for higher education (9\%) had a mean predicted
performance of only 7. For students with plans for higher education, the next 
split on mother's education level: Following the branch to the right, students 
whose mothers had 5th grade level of education or higher (74\%) had a mean 
predicted performance of 12. Following this branch to the next node of area, 
on the right branch students in urban areas (55\%) had a mean predicted
performance of 13, whereas students in rural areas (19\%) had a mean 
performance of 11. On the left branch, students whose mother's had attended 
secondary school or lower (17\%), the mean predicted performance was 10. This 
branch split next on course topic, and students in the Portugese course (12\%) 
had a mean predicted performance of 11, whereas the mean performance for 
students in the mathematics course (5\%) was 8.2. For students with no plans for 
higher education (9\%), the next node split on absences, where the mean 
predicted performance for students with no absences (5\%) was 9.1, and students 
with one or more absences, had a mean predicted performance of 4.6. 

%%%%%%%%%%%%%%%%%%%%%%%%%%%%%%%%%%%%%%%%%%%%%%%%%%%%%%%%%%%%%%%%%%%%%%%%%%%%%%%%

\begin{table*}
  \caption{Feature Importance for Random Forests Regression and 
  Gradient Boosting Model}
  \label{tab:freq}
  \begin{tabular}{lllll}
    \toprule
                        &           &  & Model & \\ 
    \midrule  
                        & Random Forests & & &  Gradient Boosting \\    
    \midrule   
    Predictor           & \% Increase MSE & & Predictor &  Relative Importance \\    
    \midrule
    Mother's Education  &  1.368 &   &      Absences	        &  16.754	\\
    Absences            &  0.884 &   &      Course	            &  11.151   \\
    Area (Urban/Rural)  &  0.592 &   &      Mother's Education  &   7.926	\\
    Higher Education    &  0.575 &   &      Age	                &   7.460   \\   
    Course              &  0.481 &   &      Go Out w/ Friends	&   5.566   \\ 
    Weekly Alcohol      &  0.473 &   &      Study Time          &   4.618   \\
    Mother's Job        &  0.463 &   &      Health	            &   3.942   \\  
    Father's Education  &  0.399 &   &      Free Time	        &   3.284	\\      
    Go Out w/ Friends   &  0.389 &   &      Family Relations 	&   3.213	\\
    Daily Alcohol       &  0.373 &   &      Weekly Alcohol      &   3.150	\\
    Age                 &  0.370 &   &      Daily Alcohol       &   2.848   \\
    School Support      &  0.347 &   &      Extra Activities    &   2.635   \\  
    Study Time          &  0.321 &   &      School Support  	&   2.550   \\
    Sex                 &  0.318 &   &      Higher Education	&   2.544   \\
    Father's Job        &  0.278 &   &      Father's Job 	    &   2.522   \\     
    Free Time           &  0.272 &   &      Guardian            &   2.124	\\
    Internet Access     &  0.204 &   &      Father's Education  &   2.059   \\   
    Family Relations    &  0.181 &   &      Family Size	        &   1.960   \\
    Extra Activities    &  0.179 &   &      Romantic Relation	&   1.889   \\
    Travel Time         &  0.122 &   &      Internet Access     &   1.875   \\ 
    Guardian            &  0.121 &   &      Sex	        	    &   1.776	\\
    Family Size         &  0.112 &   &      Family Support 	    &   1.762	\\
    Family Support      &  0.103 &   &      Travel Time	        &   1.761   \\
    Health              &  0.097 &   &      Mother's Job    	&   1.757	\\ 
    Romantic Relation   &  0.093 &   &      Paid Extra Courses  &   1.430   \\
    Paid Extra Courses  &  0.075 &   &      Area (Urban/Rural)	&   1.105	\\
    Parents' Relation   &  0.056 &   &      Parents' Relation   &   0.345   \\
    \bottomrule
  \end{tabular}	
\end{table*}
	
	
%%%%%%%%%%%%%%%%%%%%%%%%%%%%%%%%%%%%%%%%%%%%%%%%%%%%%%%%%%%%%%%%%%%%%%%%%%%%%%%%

\subsubsection{Random Forests Regression}

The mean squared error (MSE) for the random forests regression was 6.278, 
which indicates better performance for RF than a single decision tree. The 
random forests (RF) algorithm provides feature importance as a model summary; 
for regression, this is measured in terms of percent increase in MSE. The left 
side of Table 6 provides the feature importance for the RF regression sorted 
by percent increase in MSE. The algorithm selected mother's education as the 
most informative feature for predicting student performance (final grade). In 
contrast to the single decision tree, number of absences and area (urban/rural)
were selected as the second and third most important features in the model. 
Plans for higher education and course subject were also among the most 
influential predictor variables in the random forest model, but these
variables were not given as prominent a position as in the single tree. 

%%%%%%%%%%%%%%%%%%%%%%%%%%%%%%%%%%%%%%%%%%%%%%%%%%%%%%%%%%%%%%%%%%%%%%%%%%%%%%%%

\subsubsection{Gradient Boosted Regression}

The mean squared error (MSE) for the gradient boosted regression tree model  
was 18.124. Feature importance for the gradient boosted regression trees is 
presented on the right side of Table 6. Absences and course subject were 
selected as the two most important features for predicting student performance. 
The algorithm selected mother's education level, student age, going out with 
friends, and weekly study time as the next most informative variables, in 
descending order of importance. Plans for higher education was not selected 
among the most important variables for predicting student performance in 
the gradient boosted model.   

%%%%%%%%%%%%%%%%%%%%%%%%%%%%%%%%%%%%%%%%%%%%%%%%%%%%%%%%%%%%%%%%%%%%%%%%%%%%%%%%

\section{DISCUSSION}

The various models identified many of the same variables as important for 
predicting student performance, including course subject, mother's education, 
plans for higher education, weekly study time, school support, absences, 
going out with friends, and romantic relationships. Overall, students in the 
Portugese course performed better than students in mathematics. The models 
differed, however, in the number, weighting, order, and importance of the 
selected predictors. With linear regression, the predicted outcome is based
on the weighted average or combination of all the predictor variables. The 
OLS linear regression identified thirteen regressor variables in the training 
set, of which, only six were significant predictors of student performance in 
the test set, which indicates overfitting. Mother's education level, students' 
plans for higher education, and weekly study time were associated with increased 
student performance, whereas course subject, school support, and romantic 
relationships were related to decreased performance. Ridge regression reduced 
model error by shrinking the coefficient estimates towards zero, but with 
slightly higher bias and risk of underfitting. The lasso model performed variable 
selection by shrinking the coefficient values of seven non-relevant predictors 
to exactly zero, which yielded a model with a subset of twenty features. 
Although the lasso model was simpler than ridge regression, the final OLS model 
with thirteen predictor variables was more parsimonious. 

%%%%%%%%%%%%%%%%%%%%%%%%%%%%%%%%%%%%%%%%%%%%%%%%%%%%%%%%%%%%%%%%%%%%%%%%%%%%%%%%

The simple regression tree model selected course subject as the root node in 
the training set, and branched on mother's education, weekly study time, 
absences, mother's job, and student age as key nodes for predicting student 
performance. In the testing set, plans for higher education was selected as 
the root node, which may indicate overfitting, and the algorithm branched on
mother's education, residence area (rural versus urban), course subject, and 
student absences as key nodes. The single regression tree provides a simple 
model that is easily interpreted and gives the mean predicted value of 
student performance at each node. Predicted performance was highest among 
students with plans for higher education, whose mothers had attained at 
least a 5th grade education, from urban areas. The lowest predicted 
performance was found among students with no plans for higher education 
who had one or more absences. Random forests regression corrected for 
overfitting by constructing many trees and averaging across the predicted 
values. Feature importance generated by the random forests regression 
identified mother's education as the most informative variable based on the 
percent change in mean squared error (MSE), followed (in descending order) by 
student absences, area, plans for higher education, course subject, weekly
alcohol consumption, and mother's job as among the most important variables
for predicting student performance. The gradient boosted regression model 
had a higher error than for the random forests regression, which was surprising, 
given that the gradient boosted model generates a large number of short trees 
(i.e., stumps) and tries to correct for mistakes in previous trees iteratively. 
Feature importance for the gradient boosted model selected student absences 
as the most important feature, followed by course subject, mother's education, 
student age, going out with friends, and study time.

%%%%%%%%%%%%%%%%%%%%%%%%%%%%%%%%%%%%%%%%%%%%%%%%%%%%%%%%%%%%%%%%%%%%%%%%%%%%%%%%

\subsubsection{Limitations}

This study compared different methods for identifying and selecting variables 
important for predicting student performance. A limitation is that variable 
importance is not a well defined concept and lacks a theoretically based 
quantitative metric \cite{gromping09}. The linear regression model uses 
significance tests to select the variables that best predict the target outcome 
and non-significant regressors are excluded from the final model. In this sense,
the t-value provides a general measure of the importance of a given predictor. 
With random forests, node purity measures branch homogeneity for classification 
tasks, whereas MSE reduction is used for variable selection on regression tasks. 
Variable importance with random forests is affected by the number of categories
and scale of measurement of the predictor variables \cite{strobl07}, but does 
not provide direct indicators of the true importance of the variable. 
Furthermore, variable importance with random forests can be biased when 
predictors are measured on different scales or vary in number of categories. 
Much of the student data collected by educational institutions is measured on 
different scales. Researchers investigating feature importance with student data 
typically normalize or regularize the data by transforming variables measured 
on different scales to the same scale (M=0, SD=1). However, to facilitate 
interpretation of results, it is necessary to convert transformed variables 
back to their original scales. 

%%%%%%%%%%%%%%%%%%%%%%%%%%%%%%%%%%%%%%%%%%%%%%%%%%%%%%%%%%%%%%%%%%%%%%%%%%%%%%%%

\subsubsection{EDM and Learning Analytics}

This study revealed several demographic characteristics (e.g., plans for 
higher education, weekly study time, romantic relationships) that play an
important role in predicting student performance. The reliability of self-
report survey data is a potential concern. It can be difficult to obtain 
student information from educational institutions owing to privacy protections 
and confidentiality of student data. This study used archived data from a 
public repository (UCI-MLR) that included information from both school records 
a student survey. To avoid potential bias of self-reports, assessing student 
behaviour from data collected online can provide a more reliable measure of
performance. Education data mining (EDM) and learning analytics (LA) developed 
out of the increase in big data in education and the shift toward online 
learning. Much LA/EDM research data is collected within a learning management 
system (LMS), virtual learning environment (VLE), or massive open online 
course (MOOC). The Course Signals program was one of the most widely known 
platforms for tracking students at risk of falling behind using statistical 
analysis of LMS data \cite{arnoldPistilli12}. In addition to grades, course
signals combined student demographic information, academic history, and student 
interaction on the Blackboard LMS to track performance. A predictive algorithm 
was used to calculated the likelihood of student success based on performance, 
effort, history, and student characteristics. Course signals provided students 
with real-time feedback about their status in the LMS as traffic indicators 
(i.e., red, yellow, green). The assessment allowed instructors to enact 
interventions for high risk students via emails, texts, or face to face 
meetings with referrals to academic advising and academic resources center.
Courses that implemented the Signals program and provided feedback showed an 
increase in satisfactory grades, decrease in withdrawals, and improved retention. 

%%%%%%%%%%%%%%%%%%%%%%%%%%%%%%%%%%%%%%%%%%%%%%%%%%%%%%%%%%%%%%%%%%%%%%%%%%%%%%%%

\subsubsection{Student Performance, Affect, and Motivation}

Learning is a complex phenomenon that is not always directly observable and 
often inferred from behavior. In addition to quantitative measures of online
activities, ideally, a meaningful analytics system could include qualitative
measures of students' affective states (i.e., boredom, frustration, confusion) 
or motivation to model engagement in learning activities 
\cite{baker14, pardos14}. In addition, LA/EDM researchers have investigated
interactions among learners \cite{dawson14}. Theories of Social learning have 
demonstrated the importance of collaboration in learning \cite{vygotsky78}. 
From a social constructivist perspective, knowledge is constructed through 
interaction with more knowledgeable partners, including parents, siblings, 
teachers, or peers. Sociological research has also investigated the structure 
of connections in social networks \cite{granovetter73}. Social network analysis
(SNA) provides a useful methodology to explore the role of collaboration in 
learning and visualize connections among learners \cite{siemens13}.
Past research has also revealed that individual differences in students
metacognitive abilities (i.e., self-awareness, self-reflection), disposition, 
experience, and motivation are influential for developing learning relationships 
\cite{gasevic15 lang17,lester19, papamitsiou14}. Additional research would help 
to clarify the relationships among variables that predict student performance. 

%%%%%%%%%%%%%%%%%%%%%%%%%%%%%%%%%%%%%%%%%%%%%%%%%%%%%%%%%%%%%%%%%%%%%%%%%%%%%%%%

\section{Conclusion}

Predictive modeling offers a set of analytic tools for detecting patterns in 
education data and understanding the factors that contribute to successful
learning. This study compared six supervised learning models of student 
performance that varied in complexity and interpretability. Simpler models 
provided solutions that were easy to interpret but prone to overfitting;
more complex models reduced overfitting and decreased model error, but 
provided solutions that were more difficult to interpret. A general conclusion 
is that comparing the results of several models provides a more complete 
picture of factors that contribute to student performance than examining any 
single model. LA/EDM research is increasingly focused on learning outcomes in 
online platforms. Merging data from LMS or online learning platforms with 
institutional data about student demographic characteristics can provide a 
better understanding of student learning and the conditions in which it 
occurs \cite{hora19, siemens12}. Past history of student absences or failures 
are strong indicators of future performance; however, a student's future 
orientation to pursue higher education can reveal a great deal about his 
or her motivation to succeed. Using a data-driven approach to learning 
analytics based on student profiles and LMS activity can also facilitate early 
detection of at-risk students. These findings may inform decision making and 
policy efforts to address student retention and direct resources to improve successful learning outcomes.

%\cite{dekkar09}

%%%%%%%%%%%%%%%%%%%%%%%%%%%%%%%%%%%%%%%%%%%%%%%%%%%%%%%%%%%%%%%%%%%%%%%%%%%%%%%%
\begin{acks}

Thanks to Michael Smith at ICF for an providing an introduction to the field
of Learning Analytics. Thanks also to Elizabeth Whynott and Douglas Wilbur 
for helpful comments. 

\end{acks}

%%%%%%%%%%%%%%%%%%%%%%%%%%%%%%%%%%%%%%%%%%%%%%%%%%%%%%%%%%%%%%%%%%%%%%%%%%%%%%%%

\bibliographystyle{unsrt} %%ACM-Reference-Format%%
\bibliography{report} 


\end{document} 
